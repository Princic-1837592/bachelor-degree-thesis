\documentclass{article}
\usepackage{amsfonts,amssymb,amsmath}
\usepackage{float}
\pagestyle{empty}
\parindent 0pt

\begin{document}

the distrubutive property states that
$a(b+c) = ab+ac $,$  \forall a,b,c \in \mathbb{R}$\\
\$11.50 $\$11.50$\\
$$2\left(\frac{1}{x^2-1}\right)$$\\
$$2\left(\frac{1}{x^2-1}\right($$\\
$$2\left.\frac{1}{x^2-1}\right)$$\\
$$\left(\frac{1}{1+\left(\frac{1}{1+x}\right)}\right)$$


Tables:\\

\begin{tabular}{|c||c|c|c|c|c|}
\hline $x$&1&2&3&4&5\\\hline
$f(x)$&5&5&5&5&5\\\hline
\end{tabular}

\begin{table}[H]
\def\arraystretch{1.5}
\centering
\begin{tabular}{|c||c|c|c|c|c|}
\hline $x$&1&2&3&4&5\\\hline
$f(x)$&$\frac{1}{2}$&5&5&5&5\\\hline
\end{tabular}
\caption{tabella di prova f(x)}
\end{table}



Arrays:\\

\begin{align}
5x^2-9=x+3\\
5x^2-x-12=0\\
=12+x-5x^2
\end{align}


\begin{align*}
5x^2-9&=x+3\\
5x^2-x-12&=0\\
&=12+x-5x^2
\end{align*}

\newpage
\begin{enumerate}
\item pencil
\item ruler
\item notebook
\begin{enumerate}
\item page1
\item page2
\begin{enumerate}
\item page1
\item page2
\begin{enumerate}
\item page1
\item page2
\end{enumerate}
\end{enumerate}
\end{enumerate}
\end{enumerate}
\end{document}